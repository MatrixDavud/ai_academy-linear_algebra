\documentclass[12pt]{article}

% --- PACKAGES ---
\usepackage[margin=1in]{geometry}
\usepackage{amsmath}
\usepackage{amssymb}
\usepackage{amsfonts}
\usepackage{graphicx}
\usepackage{hyperref}
\usepackage{xcolor}
\usepackage{listings}

% --- HYPERREF SETUP ---
\hypersetup{
    colorlinks=true,
    linkcolor=blue,
    filecolor=magenta,      
    urlcolor=cyan,
    pdftitle={Math4AI: Programming Assignment 2},
    pdfpagemode=FullScreen,
}

% --- LISTINGS (CODE) SETUP ---
\definecolor{codegreen}{rgb}{0,0.6,0}
\definecolor{codegray}{rgb}{0.5,0.5,0.5}
\definecolor{codepurple}{rgb}{0.58,0,0.82}
\definecolor{backcolour}{rgb}{0.95,0.95,0.92}

\lstdefinestyle{mystyle}{
    backgroundcolor=\color{backcolour},   
    commentstyle=\color{codegreen},
    keywordstyle=\color{magenta},
    numberstyle=\tiny\color{codegray},
    stringstyle=\color{codepurple},
    basicstyle=\ttfamily\footnotesize,
    breakatwhitespace=false,         
    breaklines=true,                 
    captionpos=b,                    
    keepspaces=true,                 
    numbers=left,                    
    numbersep=5pt,                   
    showspaces=false,                
    showstringspaces=false,
    showtabs=false,                  
    tabsize=2
}
\lstset{style=mystyle}

% --- DOCUMENT TITLE ---
\title{\textbf{Math4AI: Linear Algebra \\ Programming Assignment 2} \\ \large \vspace{5 mm}Systems of Linear Equations \& Model Fitting}
\author{National AI Academy}
\date{\today}

% --- BEGIN DOCUMENT ---
\begin{document}

\maketitle
\hrule

\section*{Instructions}

\begin{itemize}
    \item You are required to implement several core algorithms \textbf{from scratch}. Do not use high-level library functions (like \texttt{numpy.linalg.solve}) for the core implementation of a required function.
    \item You \textbf{must} use \texttt{numpy} for verification of your results. This is a crucial step to help you debug your own code.
    \item  This is an individual assignment. Any form of academic dishonesty, including plagiarism or unauthorized assistance, will be thoroughly investigated. Proven violations will result in serious disciplinary action.
    \item For each problem, provide your Python code and the output that demonstrates its correctness and verifies it against \texttt{numpy} where required.
    \item \textbf{Generality of Functions:} Your implementations should be robust. For vector and matrix operations, write your functions to handle inputs of arbitrary (but compatible) dimensions. For instance, a dot product function should work for 2D vectors or 100D vectors without modification.
    \item \textbf{Notation:} Matrices are denoted by uppercase bold letters (e.g., $\mathbf{A}$), vectors by lowercase bold letters (e.g., $\mathbf{x}$), and scalars by lowercase italic letters (e.g., $c$).
\end{itemize}

\hrule\vspace{1em}

\newpage
\hrule
\section*{Systems of Linear Equations \& Model Fitting}
\hrule\vspace{0.5em}

\subsection*{The AI Connection: Linear Models and Direct Solutions}
Solving $\mathbf{A}\mathbf{x} = \mathbf{b}$ is the basis for \textbf{linear models}. Imagine $\mathbf{A}$ is a matrix where each row represents a customer and each column is a feature (e.g., 'time spent on site', 'number of ads clicked'). Let $\mathbf{b}$ be the total amount each customer spent. The solution vector $\mathbf{x}$ would contain the "weights" or importance of each feature. By solving for $\mathbf{x}$, we are "fitting" a model that can predict spending based on user behavior. While modern AI uses more complex methods, understanding this direct solution is the first step.


Consider the system:
$$ \mathbf{A} = \begin{pmatrix} 2 & 1 & 3 \\ 4 & 4 & 7 \\ 2 & 5 & 9 \end{pmatrix}, \quad \mathbf{x} = \begin{pmatrix} \text{weight}_1 \\ \text{weight}_2 \\ \text{weight}_3 \end{pmatrix}, \quad \mathbf{b} = \begin{pmatrix} 1 \\ 1 \\ 3 \end{pmatrix} $$

\subsection*{2.1: Gaussian Elimination from Scratch}
\textbf{Implement from scratch} a Python function \texttt{gaussian\_elimination(A, b)}.
\begin{itemize}
    \item The function should take a matrix $\mathbf{A}$ and a vector $\mathbf{b}$ as inputs.
    \item \textbf{Forward Elimination:} Convert the augmented matrix $[\mathbf{A}|\mathbf{b}]$ into row echelon form. Your implementation should handle potential division by zero by performing row swaps (pivoting).
    \item \textbf{Back Substitution:} Solve for $\mathbf{x}$ using the resulting upper triangular matrix.
    \item The function \textbf{should} handle all cases:
\begin{itemize}
    \item Return the solution vector $\mathbf{x}$ if there is exactly one \textbf{unique solution} (consistent system).
    \item Return \texttt{"No solution} if there is \textbf{no solution} (system is inconsistent).
    \item Return a clear indication (e.g., the string \texttt{"Infinite solutions"}) if there are \textbf{infinitely many solutions} (the system is still consistent but with free variables).
\end{itemize}


    \item Test your function on the system above.
\end{itemize}

\subsection*{2.2: NumPy Verification}
Use \texttt{numpy.linalg.solve(A, b)} to solve the same system and verify your implementation's correctness.

\vspace{2em}
\hrule
\section*{Submission}

You are required to submit a single ZIP file containing **two separate documents**:

\begin{enumerate}
    \item \textbf{Python File (`.py`):} This file must contain all your Python code and the verification checks. Ensure your code is well-commented.
    
    \item \textbf{Explanation Document (`.pdf`):} A separate PDF file where you provide a detailed, written explanation for each function you implemented from scratch. For each function, you should:
    \begin{itemize}
        \item Explain the algorithm you chose in your own words.
        \item Describe the key steps of your implementation.
        \item Justify any important design choices you made (e.g., how you handled certain data structures or edge cases).
    \end{itemize}
\end{enumerate}

This dual submission is designed to assess both your coding ability and your conceptual understanding of the algorithms.
\end{document}
% --- END DOCUMENT ---