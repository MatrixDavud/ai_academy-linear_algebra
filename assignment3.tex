\documentclass[12pt]{article}

% --- PACKAGES ---
\usepackage[margin=1in]{geometry}
\usepackage{amsmath}
\usepackage{amssymb}
\usepackage{amsfonts}
\usepackage{graphicx}
\usepackage{hyperref}
\usepackage{xcolor}
\usepackage{listings}

% --- HYPERREF SETUP ---
\hypersetup{
    colorlinks=true,
    linkcolor=blue,
    filecolor=magenta,      
    urlcolor=cyan,
    pdftitle={Math4AI: Programming Assignment 3},
    pdfpagemode=FullScreen,
}

% --- LISTINGS (CODE) SETUP ---
\definecolor{codegreen}{rgb}{0,0.6,0}
\definecolor{codegray}{rgb}{0.5,0.5,0.5}
\definecolor{codepurple}{rgb}{0.58,0,0.82}
\definecolor{backcolour}{rgb}{0.95,0.95,0.92}

\lstdefinestyle{mystyle}{
    backgroundcolor=\color{backcolour},   
    commentstyle=\color{codegreen},
    keywordstyle=\color{magenta},
    numberstyle=\tiny\color{codegray},
    stringstyle=\color{codepurple},
    basicstyle=\ttfamily\footnotesize,
    breakatwhitespace=false,         
    breaklines=true,                 
    captionpos=b,                    
    keepspaces=true,                 
    numbers=left,                    
    numbersep=5pt,                   
    showspaces=false,                
    showstringspaces=false,
    showtabs=false,                  
    tabsize=2
}
\lstset{style=mystyle}

% --- DOCUMENT TITLE ---
\title{\textbf{Math4AI: Linear Algebra \\ Programming Assignment 3} \\ \large \vspace{5 mm}Matrix Inverse, Decompositions \& The Normal Equations}
\author{National AI Academy}
\date{\today}

% --- BEGIN DOCUMENT ---
\begin{document}

\maketitle
\hrule

\section*{Instructions}

\begin{itemize}
    \item You are required to implement several core algorithms \textbf{from scratch}. Do not use high-level library functions (like \texttt{numpy.linalg.solve}) for the core implementation of a required function.
    \item You \textbf{must} use \texttt{numpy} for verification of your results. This is a crucial step to help you debug your own code.
    \item  This is an individual assignment. Any form of academic dishonesty, including plagiarism or unauthorized assistance, will be thoroughly investigated. Proven violations will result in serious disciplinary action.
    \item For each problem, provide your Python code and the output that demonstrates its correctness and verifies it against \texttt{numpy} where required.
    \item \textbf{Generality of Functions:} Your implementations should be robust. For vector and matrix operations, write your functions to handle inputs of arbitrary (but compatible) dimensions. For instance, a dot product function should work for 2D vectors or 100D vectors without modification.
    \item \textbf{Notation:} Matrices are denoted by uppercase bold letters (e.g., $\mathbf{A}$), vectors by lowercase bold letters (e.g., $\mathbf{x}$), and scalars by lowercase italic letters (e.g., $c$).
\end{itemize}

\hrule\vspace{1em}

\newpage
\hrule
\section*{Matrix Inverse, Decompositions \& The Normal Equations}
\hrule\vspace{0.5em}

\subsection*{The AI Connection: The Normal Equation for Regression}
In machine learning, a cornerstone of linear regression is the \textbf{Normal Equation}, which gives a closed-form solution for the optimal model weights ($\boldsymbol{\theta}$):
$$ \boldsymbol{\theta} = (\mathbf{X}^T \mathbf{X})^{-1} \mathbf{X}^T \mathbf{y} $$
Notice the \textbf{matrix inverse}! While computationally we prefer other methods, this equation shows the theoretical importance of the inverse. Decompositions like LU are workhorse algorithms that make solving such systems computationally feasible by breaking a hard problem ($\mathbf{A}^{-1}$) into simpler steps ($\mathbf{L}$ and $\mathbf{U}$).

We'll use matrix $\mathbf{A}$ from Assignment 2:
$$ \mathbf{A} = \begin{pmatrix} 2 & 1 & 3 \\ 4 & 4 & 7 \\ 2 & 5 & 9 \end{pmatrix} $$

\subsection*{3.1: Matrix Inverse via Gauss-Jordan Elimination from Scratch}
\textbf{Implement from scratch} a Python function \texttt{invert\_matrix(A)}.
\begin{itemize}
    \item The function should take a square matrix $\mathbf{A}$.
    \item Create the augmented matrix $[\mathbf{A}|\mathbf{I}]$, where $\mathbf{I}$ is the identity matrix of the same size.
    \item Apply Gauss-Jordan elimination to transform $[\mathbf{A}|\mathbf{I}]$ into $[\mathbf{I}|\mathbf{A}^{-1}]$.
    \item The function should return the inverted matrix $\mathbf{A}^{-1}$. If the matrix is singular (non-invertible), it should return \texttt{None}.
    \item Test your function on matrix $\mathbf{A}$.
\end{itemize}

\subsection*{3.2: LU Decomposition from Scratch}
\textbf{Implement from scratch} a Python function \texttt{lu\_decomposition(A)} that performs LU decomposition using the Doolittle algorithm.
\begin{itemize}
    \item The function should take a square matrix $\mathbf{A}$.
    \item It should return two matrices, $\mathbf{L}$ (lower triangular with ones on the diagonal) and $\mathbf{U}$ (upper triangular).
    \item Test your function on matrix $\mathbf{A}$.
\end{itemize}

\subsection*{3.3: NumPy Verification}
\begin{enumerate}
    \item Use \texttt{numpy.linalg.inv(A)} to compute the inverse of $\mathbf{A}$ and verify your result from 3.1.
    \item Verify your LU decomposition by computing the matrix product $\mathbf{L}\mathbf{U}$ and showing that it is equal (or very close, due to floating-point arithmetic) to the original matrix $\mathbf{A}$. Note: SciPy's \texttt{scipy.linalg.lu} is the standard library function for this.
\end{enumerate}

\vspace{2em}
\hrule
\section*{Submission}

You are required to submit a single ZIP file containing **two separate documents**:

\begin{enumerate}
    \item \textbf{Python File (`.py`):} This file must contain all your Python code and the verification checks. Ensure your code is well-commented.
    
    \item \textbf{Explanation Document (`.pdf`):} A separate PDF file where you provide a detailed, written explanation for each function you implemented from scratch. For each function, you should:
    \begin{itemize}
        \item Explain the algorithm you chose in your own words.
        \item Describe the key steps of your implementation.
        \item Justify any important design choices you made (e.g., how you handled certain data structures or edge cases).
    \end{itemize}
\end{enumerate}

This dual submission is designed to assess both your coding ability and your conceptual understanding of the algorithms.

\end{document}
% --- END DOCUMENT ---