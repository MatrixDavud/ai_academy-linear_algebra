\documentclass[12pt]{article}

% --- PACKAGES ---
\usepackage[margin=1in]{geometry}
\usepackage{amsmath}
\usepackage{amssymb}
\usepackage{amsfonts}
\usepackage{graphicx}
\usepackage{hyperref}
\usepackage{xcolor}
\usepackage{listings}

% --- HYPERREF SETUP ---
\hypersetup{
    colorlinks=true,
    linkcolor=blue,
    filecolor=magenta,      
    urlcolor=cyan,
    pdftitle={Math4AI: Programming Assignment 1},
    pdfpagemode=FullScreen,
}

% --- LISTINGS (CODE) SETUP ---
\definecolor{codegreen}{rgb}{0,0.6,0}
\definecolor{codegray}{rgb}{0.5,0.5,0.5}
\definecolor{codepurple}{rgb}{0.58,0,0.82}
\definecolor{backcolour}{rgb}{0.95,0.95,0.92}

\lstdefinestyle{mystyle}{
    backgroundcolor=\color{backcolour},   
    commentstyle=\color{codegreen},
    keywordstyle=\color{magenta},
    numberstyle=\tiny\color{codegray},
    stringstyle=\color{codepurple},
    basicstyle=\ttfamily\footnotesize,
    breakatwhitespace=false,         
    breaklines=true,                 
    captionpos=b,                    
    keepspaces=true,                 
    numbers=left,                    
    numbersep=5pt,                   
    showspaces=false,                
    showstringspaces=false,
    showtabs=false,                  
    tabsize=2
}
\lstset{style=mystyle}

% --- DOCUMENT TITLE ---
\title{\textbf{Math4AI: Linear Algebra \\ Programming Assignment 1} \\ \large \vspace{5 mm}Vector Operations \& Semantic Similarity}
\author{National AI Academy}
\date{\today}

% --- BEGIN DOCUMENT ---
\begin{document}

\maketitle
\hrule

\section*{Instructions}

\begin{itemize}
    \item You \textbf{must} use \texttt{numpy} for verification of your results. This is a crucial step to help you debug your own code.
    \item  This is an individual assignment. Any form of academic dishonesty, including plagiarism or unauthorized assistance, will be thoroughly investigated. Proven violations will result in serious disciplinary action.
    \item For each problem, provide your Python code and the output that demonstrates its correctness and verifies it against \texttt{numpy} where required.
    \item \textbf{Generality of Functions:} Your implementations should be robust. For vector and matrix operations, write your functions to handle inputs of arbitrary (but compatible) dimensions. For instance, a dot product function should work for 2D vectors or 100D vectors without modification.
    \item \textbf{Notation:} Matrices are denoted by uppercase bold letters (e.g., $\mathbf{A}$), vectors by lowercase bold letters (e.g., $\mathbf{x}$), and scalars by lowercase italic letters (e.g., $c$).
\end{itemize}

\hrule\vspace{1em}

\newpage
\hrule
\section*{Vector Operations \& Semantic Similarity}
\hrule\vspace{0.5em}

\subsection*{The AI Connection: Words as Vectors}
In Natural Language Processing (NLP), we represent words as high-dimensional numerical vectors called \textbf{word embeddings}. The revolutionary idea is that a word's meaning is captured by its relationships to other words, which can be encoded geometrically. For example, we might find that $\mathbf{v}_{\text{king}} - \mathbf{v}_{\text{man}} + \mathbf{v}_{\text{woman}} \approx \mathbf{v}_{\text{queen}}$. This shows the vector space has learned a "gender + royalty" concept. The dot product between vectors measures their semantic similarity. This entire field is built upon the fundamental vector operations you will implement here.

\vspace{1em}
\textit{Source: For further reading, see Mikolov, T., et al. (2013). "Efficient Estimation of Word Representations in Vector Space." arXiv:1301.3781.}
\vspace{1em}

We will use simplified, 3D vectors for this exercise.

\subsection*{1.1: The 'King - Man + Woman' Analogy}
Implement the vector arithmetic $\mathbf{v}_{\text{king}} - \mathbf{v}_{\text{man}} + \mathbf{v}_{\text{woman}}$ using basic Python lists and loops. Compare your result with the vector $\mathbf{v}_{\text{queen}}$.

\subsection*{1.2: Cosine Similarity from Scratch}
The cosine similarity between two vectors $\mathbf{u}$ and $\mathbf{v}$ is a measure of their orientation similarity and is defined as:
$$ \text{similarity}(\mathbf{u}, \mathbf{v}) = \cos(\theta) = \frac{\mathbf{u} \cdot \mathbf{v}}{\|\mathbf{u}\| \|\mathbf{v}\|} $$
where $\mathbf{u} \cdot \mathbf{v}$ is the dot product and $\|\mathbf{u}\|$ is the Euclidean norm (or length) of the vector, $\|\mathbf{u}\| = \sqrt{u_1^2 + u_2^2 + \dots + u_n^2}$.

\begin{enumerate}
    \item \textbf{Implement from scratch} a Python function \texttt{dot\_product(u, v)} that computes the dot product of two vectors.
    \item \textbf{Implement from scratch} a Python function \texttt{norm(u)} that computes the L2 norm of a vector.
    \item \textbf{Implement from scratch} a Python function \texttt{cosine\_similarity(u, v)} that uses your two functions above.
    \item Calculate the cosine similarity between the result from 1.1 and $\mathbf{v}_{\text{queen}}$. A value close to 1 indicates very high similarity.
\end{enumerate}

\subsection*{1.3: NumPy Verification}
Repeat the analogy calculation and cosine similarity using NumPy array operations. Verify that your scratch implementations match the results from \texttt{np.dot()}, \texttt{np.linalg.norm()}, and NumPy-based calculations.

\vspace{2em}
\hrule
\section*{Submission}

You are required to submit a single ZIP file containing **two separate documents**:

\begin{enumerate}
    \item \textbf{Python File (`.py`):} This file must contain all your Python code and the verification checks. Ensure your code is well-commented.
    
    \item \textbf{Explanation Document (`.pdf`):} A separate PDF file where you provide a detailed, written explanation for each function you implemented from scratch. For each function, you should:
    \begin{itemize}
        \item Explain the algorithm you chose in your own words.
        \item Describe the key steps of your implementation.
        \item Justify any important design choices you made (e.g., how you handled certain data structures or edge cases).
    \end{itemize}
\end{enumerate}

This dual submission is designed to assess both your coding ability and your conceptual understanding of the algorithms.

\end{document}
% --- END DOCUMENT ---